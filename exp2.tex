\documentclass{article}

\usepackage[version=3]{mhchem} % Package for chemical equation typesetting
\usepackage{siunitx} % Provides the \SI{}{} and \si{} command for typesetting SI units
\usepackage{graphicx} % Required for the inclusion of images
\usepackage{natbib} % Required to change bibliography style to APA
\usepackage{amsmath} % Required for some math elements 
\usepackage{graphicx}
\usepackage{comment}
\usepackage{amsmath}
\usepackage{amsfonts}
\usepackage{amssymb}
\usepackage[margin=1in]{geometry}
\usepackage{textcomp}
\usepackage[section]{placeins}






\title{Study of Diode Rectifier Circuits \\ EEE-2302} % Title

\author{Mohammad \textsc{Zakaria}} % Author name

\date{\today} % Date for the report

\begin{document}
	
	\maketitle % Insert the title, author and date
	
	\begin{center}
		\begin{tabular}{l r}
			Date Performed: & January 31, 2021 \\ % Date the experiment was performed
			% Partners: & James Smith \\ % Partner names
			% & Mary Smith \\
			Instructor: & Lokman Hossain % Instructor/supervisor
		\end{tabular}
	\end{center}
	
	
	\section{Objective:}
	
	To understand principle of diode in converting ac into dc and to study different diode rectifier circuits.
	
	
	\section{Objective:}
	To understand principle of diode in converting ac into dc and to study different diode rectifier circuits.
	
	
	\section{Theory:}
	The diode rectifier converts the input sinusoidal voltage $V_s$ to a uni-polar output $V_o$. There are
	two types of rectifier circuits: (i) Half-wave rectifier and (ii) Full-wave rectifier.
	
%	\subsection{PIV (Peak Inverse Voltage)}
	\begin{description}
		\item[PIV]  is the peak inverse voltage that appears across the diode when it is reverse-biased. For half wave
		rectifier $PIV = V_m$
	\end{description}
	
%	\subsection{Ripple Factor:}
		\begin{description}
			\item[Ripple factor:]  A rectifier converts alternating currents into a unidirectional current, periodically fluctuating components
			still remaining in the output wave. A measure of the fluctuating component is given by the ripple factor r,
			which is defined as
			r = RMS value of alternating components of wave/Average value of wave
			For a half-wave rectifier, r = 1.21 and for a full wave rectifier r = 0.482
		\end{description}

		\begin{description}
			\item[Filter:] The rectifier with a filter is shown in Fig 1. When capacitor charges to Vp(12V p-p), input voltage decreases
			immediately but capacitor will not charge its voltage instantaneously. As a result diode will be reverse
			biased and stop conducting. The stored charges on the capacitor will be released through R.
		\end{description}

	
	
	\section{Equipments:}
	\begin{tabular}{ll}
		Trainer board \\
		Multimeter \\
		Resistor \\
		Capacitor 1\textmu F,\ 47\textmu F, \ 220\textmu F, 1000\textmu F \\
		Diode \ & 4 pieces \\
	\end{tabular}
	
	
	
	
	\section{Circuit Diagram:}
	
	\begin{figure}[!htb]
		\begin{center}
			\includegraphics[width=0.65\textwidth]{ckt-diagram-halfwave.pdf} % Include the image placeholder.png
			\caption{Circuit diagram for half-wave rectifier.}
		\end{center}
	\label{fig:halfwave rectifier} 	The input and output of the rectifier are drawn in fig. 1. Diode conducts only when it is forward biased. For $V_s = V_m sin\omega t$, DC voltage of a half wave rectifier is $V_{DC} = (V_m -V_T) / \pi $ ; where $V_T \approxeq 0.7 $ \\
	
	\end{figure}



\begin{figure}[!htb]
	\centering
	\includegraphics[width=0.7\linewidth]{Vout1}
	\caption{}
	\label{fig:vout1} Output voltage, when no capacitor is connected. Output is pulsating dc.
\end{figure}

\begin{figure}[!htb]
	\centering
	\includegraphics[width=0.7\linewidth]{Vout-for-1UFD-capacitor}
	\caption{}
	\label{fig:vout-for-1ufd-capacitor} Output voltage when 1\textmu F capacitor is connected in parallel with the load. Ripple is decreasing, still this is not pure dc.
\end{figure}
	
\begin{figure}[!htb]
	\centering
	\includegraphics[width=0.7\linewidth]{Vout-for-47UFD}
	\caption{}
	\label{fig:vout-for-47ufd} This is almost nearer to pure dc.
\end{figure}

	\begin{figure}[!htb]
		\centering
		\includegraphics[width=0.7\linewidth]{Vout-and-Vin-for-halfwavefinal}
		\caption{}
		\label{fig:vout-and-vin-for-halfwavefinal} Input and output voltage curve for half-wave rectifier. Here we connected 220\textmu F capacitor. From the simulation graph the output seems to pure dc.
	\end{figure}

% Full wave bridge rectifier

	
	\begin{figure}[!htb]
		\begin{center}
			\includegraphics[width=0.65\textwidth]{ckt-diagram-fullwave.pdf} % Include the image placeholder.png
			\caption{Circuit diagram for bridge rectifier.}
		\end{center}
	\end{figure}	


	\begin{figure}[!htb]
		\centering
		\includegraphics[width=0.7\linewidth]{Vout1}
		\caption{}
		\label{fig:vout1} Output voltage for full wave bridge rectifier when no capacitor is connected.
	\end{figure}
	
	\begin{figure}[!htb]
	\begin{center}
		\includegraphics[width=0.65\textwidth]{Vout1-for-bride-fullwave-with-1ufd-capacitor.pdf} % Include the image placeholder.png
		\caption{} Output voltage curve, when 1 \textmu F capacitor is connected in parallel with the load.
	\end{center}
	\end{figure}



	\begin{figure}[!htb]
	\begin{center}
		\includegraphics[width=0.65\textwidth]{Vout1-for-bride-fullwave-with-47ufd-capacitor.pdf} % Include the image placeholder.png
		\caption{} Output voltage curve, when 47 \textmu F capacitor is connected in parallel with the load.
	\end{center}
	\end{figure}
	
	
	\begin{figure}[!htb]
		\centering
		\includegraphics[width=0.7\linewidth]{Vout1-for-bride-fullwave-with-220ufd-capacitor.pdf}
		\caption{}
		\label{fig:vout1} Output voltage for full wave bridge rectifier when 220 \textmu F capacitor is connected.
	\end{figure}

\section{Report:}
Average value of the load voltages in fig 1, \
$V_{DC} = \dfrac{V_m -V_T}{\pi}$ = $ \dfrac{8-07}{3.1416} $ = 2.32V \\

Average value of the load voltage in fig 6, \ $V_{DC} = \dfrac{2(V_m -2V_T)}{\pi}$ = $ \dfrac{2(8-2 \times0.7)}{3.1416} $ = 4.20V \\

In this experiment, 1000\textmu F capacitor act as a better filter. The more the capacitance the lower the ripple factor, the pure the output voltage. In half wave rectifier the major disadvantages are high ripple factor and low efficiency. The higher the ripple factor is the lower the efficiency.

\section{Discussion:} We have learned how diode convert ac into dc in this sessional class. In our first step, we didn't found proper simulation graph for bridge rectifier output voltage. This was because of proper grounding. In bridge rectifier, we cannot find proper simulation for $V_{out}t$ and $V_{in}$ at a time. That is we can't observe output and input at a time in full wave rectifier. We also learned that upto PIV, diode won't be damage.

	
\end{document}